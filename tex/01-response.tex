%! TEX root=../main.tex

\section{Responses}
  \subsection{Response 1}
    \begin{quotation}
      Listening is something that comes easier to me than talking.
        As I indicated by my goals, one of which is to be less concise, I do not
        often just volunteer information about myself.  I prefer to listen to
        what others have been thinking about or learn more about them. I can
        relate to Erin because I would probably be wondering why I did not get
        the bonus that they had talked about, although instead of making a fuss
        about it I would probably have just checked out of the situation.
        Overall, the way Ed started the conversation, despite the fact he was
        trying to surprise her, would have caused anyone to be a bit upset.
        I do not really think that gender had anything to do with this situation
        and the reactions they had. It seems to me that this would have been
        more influenced by personality, not gender.

      I believe that we can define listening as processing information or
        emotions from others.  One article I read when studying effective and
        not effective listening skills, which talks about things effective
        listeners do states, “These behaviors include responses such as
        ``uh huh'' and ``hmmm,'' as well as other nonverbal behaviors including
        nodding, smiling, and adjusting one's posture. Other active listening
        behaviors include asking questions, making eye contact, and not
        interrupting the speaker” (Fedesco, 2015).  Asking questions can help
        you to not only better remember information, but also better understand
        it. That article goes on to say that the more you use those behaviors
        indicates how invested you are in a certain conversation. Another
        article I read makes this claim that being a good listening can offer,
        ``A greater number of friends and social networks, improved self-esteem
        and confidence, higher grades at school and in academic work, and even
        better health and general well-being'' (Listening Skills). This means
        that being a good listener can provide a large number of benefits. Being
        a good or bad listener would definitely affect your grades in school.

      As I tried to find a difference between male and female in terms of
        listening, I did not find anything that indicated one listens better
        than the other.  I read an article which states, “Despite activating
        different activity centers within the brain, genders perform equally on
        measures of cognitive function. This means that although we listen and
        assimilate information differently, the difference does not appear to
        affect cognition or our ability to listen” (McCormick,2018). Therefore,
        our brains may function in a different way, but that does not indicate
        that either men or women listen better simply based on their gender.
        I would say my research aligns pretty well with how I communicate, with
        exception to the occasionally accidental interruption.  Therefore, I
        would still classify myself as a good listener.  I often ask questions,
        make eye contact, and am sympathetic to the person speaking.

      Effective Listeners Top 5 List:
      \begin{enumerate}
        \item Asking questions
        \item Making eye contact
        \item Not interrupting
        \item Smiling
        \item Being polite
      \end{enumerate}

      Ineffective Listeners Top 5 List
      \begin{enumerate}
        \item Interrupting
        \item Looking away
        \item Crossing their arms
        \item Not reacting with smiles or frowns (not sympathetic)
        \item Often changing the subject
      \end{enumerate}
    \end{quotation}

    \paragraph{This is a response to Thora Smith on Post ID 43465667}
      Placeholder.
